
\chapter{Menu Tags}\label{tags_chapter}
\minitoc 


All vertices of different biological structures can be given a specific integer values (0, 1, 2, 3 ...), in order to identify them. Such integer arrays are referred to as "tag arrays". 
As stated earlier, a given unselected surface can be colored using the currently active tag array (identified by its "name"), if that surface contains a tag array of that name (see also Fig. \ref{4color_modes}-B p.\pageref{4color_modes}). To do so, the array display mode button must be pressed (\includegraphics[scale=0.7]{images/04/show_color_scale.png}), and a \textbf{tag} array must be selected as the currently active array (ex:\includegraphics[scale=0.5]{images/04/scalarcombo_tag.png}). The way tag arrays are translated into colors can be set up using tag maps, also referred to as "Lookup tables" (LUT) or color transfer functions. \\
Contrary to "scalar arrays" (see preceding section), tags are usually drawn manually with "painting tools". \\
For convenience purposes, as selected surfaces are drawn "grey" in MorphoDig, unselected surfaces objects can be tagged (tagging uniform "grey" objects without visual feedback would be uneasy). Tagging is the only way to modify an unselected surface in MorphoDig. \\
In order to be able to edit tags on a surface, the most common way is to create a new empty tag array for this surface (see section \ref{empty_tag_array} p. \ref{empty_tag_array} for further details). You may read section \ref{tag_starting_guide} p.\pageref{tag_starting_guide} for a quick tagging starting guide. 


\section{Open Tags window}
The "Tags" window can be also opened by clicking on "\includegraphics[scale=0.7]{images/04/tag_edit.png}" (see Fig. \ref{tags_window}). The Tags window contains most options related to tags. The active tag array can be chosen here, as well as the active tag map. Tag maps (name, color, opacity of different biological structures) can be defined and modified here as well. The currently used tag tool is also chosen in this window (pencil "\includegraphics[scale=0.7]{images/12/pencil.png}", paint bucket "\includegraphics[scale=0.7]{images/12/paint_bucket.png}" or lasso "\includegraphics[scale=0.7]{images/12/lasso.png}"). 


\begin{figure}
  \centering
  \includegraphics[scale=1]{images/12/tags_window2.png}
\caption{Tags window. This window is divided in different subsections. \textbf{1)} chose current active tag array.  \textbf{2)} the active tag array can be deleted from selected/all surface objects in this sections. \textbf{3)} chose current active tag map, which transforms integer values into color and opacity on the screen. \textbf{4)} operations on the active tag map (from left to right). a: reinitialize tag map. b: Add tag map to presets = duplicate current active tag map and create a new custom tag map. c: export tag map(s) inside a .TAG or .TGP file. d: change active color map name (only possible for custom tag maps). d: delete active color map (only possible for custom tag maps). \textbf{5)} active tag tools (from left to right): a: pencil. b: paint bucket. c: lasso tag. d: pencil radius size (in pixels rendered on the screen).  \textbf{6)} The tag map table: each line of this table is associated with an integer (exterior: 0; Tag1: 1 etc.), a color and an opacity. The active tag tools will paint a surface using the active tag. \textbf{7)} modification controls of the active tag map (from left to right). a: add a line to tag map table. b: remove last line from tag map table. c: reset active tag (merge with Tag 0 = Exterior). d: merge two tags.}	
\label{tags_window}
 \end{figure}


\noindent



\section{Create new empty tag array for each selected surface}\label{empty_tag_array}


\begin{minipage}{0.5\textwidth}
This option will create a new "empty" tag array (the name of this new array is given by the user). "Empty", in this context, means that all vertices are assigned to the "Exterior" tag (tag=0). 
\end{minipage} 

\begin{minipage}{0.5\textwidth}\centering
  \includegraphics[scale=0.5]{images/11/empty_tag_array.png}
 \captionof{figure}{Creating a new tag array}
 \end{minipage} 



\section{Create new tag array based on currently displayed colors for each selected surface}

This option (see Fig. \ref{rgb_conversion}) may be useful when you just have opened a .ply file already containing RGB colors (for instance, let us suppose that you have painted a surface using MeshLab software, and that you wish to convert those colors into tags). RGB colors contained in .ply files are inserted inside the RGB scalar when opened with MeshTools. I you plan to transform those RGB values into tags with MeshTools, be aware of the fact that RGB scalar is reinitialised extremely frequently: for instance as soon as you activate the tag or the scalar display mode, or whenever you change the objects' color rendering. So I advise you to use the present option only immediately after opening the surface.
\begin{figure}
  \centering
  \includegraphics[scale=0.5]{images/Tags/Convert.png} 
	\caption{Convert RGB scalars to TAGS window}
\label{rgb_conversion}
\end{figure}
You have two options :\\
\textbf{\underline{1) exact color match}}\\
$\rightarrow$ In that case, in order to be given a tag value other than Tag 00, a vertex must satisfy the following condition: its RGB scalar value should match one of the 25 colors defined in the ``Tags options" window. If a vertex does not satisfy this condition, it is given the Tag 00 value.\\
\textbf{\underline{2) Define tags following the first 25 distinct colors found in RGB scalar}}\\
$\rightarrow$ In that case the following sequence of operations is applied:\\
a) ISE-MeshTools searches for the first 25 distinct RGB colors inside the object; ISE-MeshTools gives them numbers ranging from 0 to 24. Important : if more than 25 distinct colors exist in the RGB scalar, they are not given a tag value.\\
b) ISE-MeshTools updates tag colors according to the 25 first colors found in the RGB scalar.\\
c) All vertices are given a Tag value following the procedure defined above in \#1 (exact color match).\\
Advantages of \#1 : if you have well prepared your RGB colors so that they matches well those associated with the 25 tags, this procedure will work perfectly.\\
Advantages of \#2 : if you have up to 25 distinct RGB colors in your .ply file and if you do not want to bother to edit manually the 25 tags and give them a RGB color, you will save time with this procedure. 
\\Drawbacks of both methods: if your file contains more than 25 distinct RGB colors, you will definitely lose information in the process.
\\Be also aware that when saving .ply files with ISE-MeshTools, the RGB coulours saved within the file are those currently rendered in the 3D screen. If you spent time to color a 3D surface in .ply format (for instance with MeshLab) and you have opened it subsequently with ISE-MeshTools and have saved it again, there is a high probability that all the initial RGB colors will have been replaced with those rendered in ISE-MeshTools.


\section{Create new tag array based on connectivity for each selected surface}

This option involves vtkPolyDataConnectivityFilter. This option will retrieve all non-connected regions of selected surfaces and assign to each of them a unique Tag id (see Fig. \ref{tag_connected}).
\begin{figure}
  \centering
  \includegraphics[scale=0.45]{images/Tags/Lemur_tag_input_output.png} 
	\caption{Left: original surface. Right: the same mesh automatically tagged into 304 non connected regions.}
\label{tag_connected}
 
\end{figure}


\section{Tagging with MorphoDig: a quick starting guide}\label{tag_starting_guide}
\textbf{\underline{}}\\
\textbf{1: Make sure that your surface contains a Tag array}: for instance, you may create a new empty tag array for this surface (see section \ref{empty_tag_array} p. \ref{empty_tag_array} for further details). Select this tag array as the currently active array.\\\\
\noindent
\textbf{2: Open the Tags window}: a: if not already active, select desired active Tag. b: chose a tag tool (pencil "\includegraphics[scale=0.7]{images/12/pencil.png}", paint bucket "\includegraphics[scale=0.7]{images/12/paint_bucket.png}" or lasso "\includegraphics[scale=0.7]{images/12/lasso.png}").  \\\\
\noindent
\textbf{3: Tag in MorphoDig's main window.} \\ a:when using the pencil "\includegraphics[scale=0.7]{images/12/pencil.png}" or the paint bucket "\includegraphics[scale=0.7]{images/12/paint_bucket.png}", you have two options:\\
 "T" pressed +left click: allows color override\\
 "T" pressed +right click: does not allow color override (see below for explanations of what color overrid is)\\
b: when using the lasso, maintain mouse left click pressed and draw a contour of the region which should be tagged.



\subsection{Pencil tag tool}
\includegraphics[scale=0.7]{images/12/pencil.png}\\
\textbf{Pencil tag tool controls:}\\
"T" pressed + left mouse click : tags the selected surface using currently active tag.\\
"T" pressed + right click : tags the selected surface using currently active tag \textbf{without color override}. Tag propagation will start at the picked vertex of a given color, but will stop when meeting another color.\\

\textbf{Pencil tag special option:}\\
\noindent
pencil tag size (in number of pixels on the screen can be modified in the Tags window.



\subsection{Paint bucket tag tool}
\includegraphics[scale=0.7]{images/pixmap/Flood_fill.png}\\
\textbf{Paint bucket tag tool controls:}\\
"T" pressed + left mouse click : tags the selected surface using currently active tag.\\
"T" pressed + right click : tags the selected surface using currently active tag \textbf{without color override}. Tag propagation will start at the picked vertex of a given color, but will stop when meeting another color.\\
 
When surfaces contain a lot of disconnected regions in space (for instance: the skeleton of a fetus), the paint bucket tool can be usefull to paint each disconnected element one by one in one mouse click. But be careful if your all regions of your surface are connected together: when using "T" + left click ("color override" allowed), as it will paint the whole object uniformly.

\subsection{Lasso tag tool} \label{lasso_tag_section}
\includegraphics[scale=0.7]{images/pixmap/Lasso_plus.png}\\
Once the "lasso tag tool" button is pressed, draw the region to be tagged by dragging the mouse on the screen while maintaining the left button pressed. Then release the left mouse button. 

\subsection{Merge tags.}
\noindent
\begin{minipage}{0.5\textwidth}
Two tagged regions can be merged into a single one. All source tags will be put into target tags. \end{minipage}    
\begin{minipage}{0.5\textwidth}\centering
  \includegraphics[scale=0.5]{images/Tags/Merge_tags.png}
 \captionof{figure}{Merge tags window}
 \end{minipage} 
\noindent
\begin{figure}
  \centering
  \includegraphics[scale=0.25]{images/Tags/Merge.png} 
	\caption{Example of tag merging. Left : cranium of \textit{Microcebus murinus} presenting the parietal region
tagged in yellow, the frontal region tagged in orange. Right : frontal tag region was merged into
the parietal region.}
\label{merge_tags}
\end{figure}
\chapter{Main window, Open data, Save data, Undo-Redo actions}
\minitoc  

\section{Main window.}
\begin{figure}
  \centering
  \includegraphics[scale=0.27]{images/03/morphodig_gui.png} 
	\caption{MorphoDig's GUI. Controls are organized around the main 3D rendering window.}
\label{gui}
 
\end{figure}

MorphoDig's main window is organized as shown in Fig. \ref{gui}. The most commonly used controls are directly accessible nearby the main 3D rendering window, and are sorted by function. 

 \section{Open and save data}


\begin{itemize}
\item There are three main ways to open data in MorphoDig. First, you can open specific files via the menu "File". You may also open data by clicking on the "open" (\includegraphics[scale=0.03]{images/03/open_data.png}) button inside the main window. You may also drag and drop files within the main 3D window.  
STL, PLY and VTK polydata surface files can be opened by MorphoDig, and many files specific to that software (files, curves, positions, color maps, tag maps, orientation-helper labels etc.).
\item In most cases, only selected objects (or some of their properties) can be saved; most "save" functionalities will not affect unselected objects. There are two main ways to save data in MorphoDig. First, you can save specific files through the menu "File". You may also save a project with the button "save project" (\includegraphics[scale=0.03]{images/03/save_data.png})  inside the main window. Only selected objects will be considered when saving a project.
\end{itemize}





\section{Undo-Redo actions}

\begin{itemize}
\item Most actions, such as opening data, deleting data, selecting/unselecting objects, placing landmarks, modify the position of landmarks/surfaces, surface tagging etc. can be undone by clicking the "undo" button (\includegraphics[scale=0.05]{images/03/undo.png}).
\item The same actions can be redone by clicking on the "redo" button (\includegraphics[scale=0.05]{images/03/redo.png}).
\end{itemize}


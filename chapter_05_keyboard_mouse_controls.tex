\chapter{Keyboard and mouse controls}
\minitoc  

 \section{Keyboard and mouse controls}
\rowcolors{2}{}{gray!25}
\begin{tabularx}{\linewidth}{ | c | X | }
 \hline			
   Ctrl+A & Selects all objects \\ \hline				
   Ctrl+D & Unselects all objects \\ \hline				
   L+ left click & Creates a landmark (either ``normal", ``target", ``curve node", ``curve handle" or ``flag" landmark). \\ \hline			
L + right click & If a single landmark is selected, its position is
moved. Nothing happens if no landmark is selected
or if more than one landmark are selected \\ \hline			

Left mouse button drag 
& Camera mode : camera rotation.\newline
 Object mode : object rotation.\newline
Landmark mode : camera rotation. \\ \hline			

Ctrl + left mouse button drag 
& Camera mode : object rotation.\newline
 Object mode : camera rotation.\newline
 Landmark mode : object rotation. \\ \hline	
		
Right mouse button drag & Draws a yellow rectangle. Once right button is
released, all objects (surfaces and landmarks)
falling inside the rectangle get selected/unselected,
depending on their initial selection status \\ \hline	
		

Middle mouse button roll (roll wheel) & Zoom / unzoom \\ \hline		
	
Middle mouse button drag 
& Camera mode : camera translation\newline
 Object mode : object translation\newline
 Landmark mode : camera translation \\ \hline	
		
Ctrl + middle mouse button drag 
& Camera mode : object translation\newline
Object mode : camera translation\newline
Landmark mode : object translation \\ \hline			

« Del » & All selected objects are deleted \\ \hline			

T + left click & Picks a vertex and tags corresponding surface with active tag value (a tag array must be active, and tag mode must be activated)  \\ \hline			
 
T + right click & Picks a vertex and tags corresponding surface with active tag value (a tag array must be active, and tag mode must be activated). The only potentially affected vertices are those having initially the same color as that of the picked vertex.   \\ \hline			

 \end{tabularx}

\section{Additional controls}
Additional controls are available when using ``lasso cut" or ``lasso tag" (lasso mode should be active):\\
\begin{tabularx}{\linewidth}{ | c | X | }
\hline			
Left mouse click and drag & Draws  the lasso polygon contour\\ \hline			

Left mouse release &  all the vertices falling within the polygonal region (outside or inside the polygon) are either:\newline
- given the color corresponding the active tag (lasso tag)\newline
- inserted into a new object (lasso cut inside).\newline 
- deleted from the new object (lasso cut outside).\newline 
See lasso cut (section \ref{lasso_cut_section}) and lasso tag (section \ref{lasso_tag_section}) sections for further information.\\ \hline	
		

\end{tabularx}


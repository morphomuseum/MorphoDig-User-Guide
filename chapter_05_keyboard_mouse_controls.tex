\chapter{Keyboard and mouse controls}
\minitoc  

 \section{Keyboard and mouse controls}
\rowcolors{2}{}{gray!25}
\begin{tabularx}{\linewidth}{ | c | X | }
 \hline			
   Ctrl+A & Selects all objects \\ \hline				
   Ctrl+D & Unselects all objects \\ \hline				
	 Ctrl+Z & Undo last action \\ \hline				
	Ctrl+Y & Redo last action \\ \hline				
   L+ left click & Creates a landmark (either ``normal", ``target", ``curve node", ``curve handle" or ``flag" landmark). Warning: this option does not work with touchpads, you need a "real" mouse to be able to combine "T" and a left click (see below for an alternative).\\ \hline	

   
 Shift+ left click & In curve node digitation mode :  creates a new curve starting node. Otherwise, creates a landmark (either ``normal", ``target",  ``curve handle" or ``flag" landmark). Works with touchpads\\ \hline	   

 J+ left click & In curve node digitation mode :  creates a new curve milestone node. Otherwise, creates a landmark (either ``normal", ``target",  ``curve handle" or ``flag" landmark). \\ \hline	      		

 K+ left click & In curve node digitation mode :  creates a new curve node that will be connected to the preceding starting node. Otherwise, creates a landmark (either ``normal", ``target",  ``curve handle" or ``flag" landmark). \\ \hline	      		
		
L + right click & If a single landmark or flag is selected, its position is
moved. Nothing happens if no landmark nor flag is selected
or if more than one landmark or flag are selected. Warning: this option does not work with touchpads, you need a "real" mouse to be able to combine "T" and a right click. \\ \hline			
Shift + right click & Same as above, but works with touchpads.\\ \hline			

Left mouse button drag 
& Camera mode : camera rotation.\newline
 Object mode : object rotation.\newline
Landmark mode : camera rotation. \\ \hline			

Ctrl + left mouse button drag 
& Camera mode : object rotation.\newline
 Object mode : camera rotation.\newline
 Landmark mode : object rotation. \\ \hline	
		
Right mouse button drag & Draws a yellow rectangle. Once right button is
released, all objects (surfaces and landmarks)
falling inside the rectangle get selected/unselected,
depending on their initial selection status \\ \hline	
		

Middle mouse button roll (roll wheel) & Zoom / unzoom \\ \hline		
	
Middle mouse button drag 
& Camera mode : camera translation\newline
 Object mode : object translation\newline
 Landmark mode : camera translation \\ \hline	
		
Ctrl + middle mouse button drag 
& Camera mode : object translation\newline
Object mode : camera translation\newline
Landmark mode : object translation \\ \hline			

« Del » & All selected objects are deleted \\ \hline			



 \end{tabularx}

\begin{tabularx}{\linewidth}{ | c | X | }
T + left click (surfaces) & Picks a vertex and tags corresponding surface with active tag value (a tag array must be active, and tag mode must be activated). Warning: this option does not work with touchpads, you need a "real" mouse to be able to combine "T" and a left click (no alternative yet).  \\ \hline			
T + left click (volumes) & Only works when masking is enabled. Picks a voxel and masks corresponding volume region (using either a sphere or a cylindric shape) Warning: this option does not work with touchpads, you need a "real" mouse to be able to combine "T" and a left click (no alternative yet).  \\ \hline			 

T + right click & Picks a vertex and tags corresponding surface with active tag value (a tag array must be active, and tag mode must be activated). The only potentially affected vertices are those having initially the same color as that of the picked vertex. Warning: this option does not work with touchpads, you need a "real" mouse to be able to combine "T" and a right click (no alternative yet).  \\ \hline			

T + middle mouse roll & Increases or decreases pencil radius (tags and masks).  \\ \hline			
\end{tabularx}

\section{Additional controls}
Additional controls are available when using ``lasso cut",  ``lasso tag",  ``lasso mask"  (lasso mode should be active) or "rubber band cut", "rubber band tag", "rubber band mask" (rubber band mode should be active):\\
\begin{tabularx}{\linewidth}{ | c | X | }
\hline			
Left mouse click and drag & Draws the lasso polygon/rectangular contour\\ \hline			

Left mouse release (surfaces)&  all the vertices falling within the polygonal/rectangular region (outside or inside the polygon/rectangle) are either:\newline
- given the color corresponding the active tag (lasso/rubber band tag)\newline
- inserted into a new object (lasso/rubber band cut inside).\newline 
- deleted from the new object (lasso/rubber band cut outside).\newline 
See lasso or rubber band cut (sections \ref{lasso_cut_section} p.\pageref{lasso_cut_section} and \ref{rubber_band_cut_section} p.\pageref{rubber_band_cut_section}) and lasso tag or rubber band tag (sections \ref{lasso_tag_section} p.\pageref{lasso_tag_section} and \ref{rubber_band_tag_section} p.\pageref{rubber_band_tag_section}) sections for further information.\\ \hline	
		
Left mouse release (volumes)&  all the voxels falling within the polygonal/rectangular region (outside or inside the polygon/rectangle) are either masked or unmasked, depending on the options used. See section "volume rendering masking"' of chapter \ref{volumes_chapter}  p.\pageref{volume_rendering_masking}  for further information.\\ \hline	
\end{tabularx}


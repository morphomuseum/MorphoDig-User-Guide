
 \chapter{F.A.Q.}
		\minitoc  
    \section{How should I cite MorphoDig in scientific publications ?}
    You may  cite MorphoDig with the following reference :\\
		Lebrun, R. MorphoDig, an open-source 3D freeware dedicated to biology. IPC5, Paris, France; 07/2018.
    \section{Is MorphoDig a geometric morphometrics software ?}
    No. However, you can digitize 3D landmarks on complex 3D surfaces or on 3D volume rendering representations using MorphoDig, which you 
		can use in other software.
		\section{Can I produce/extract 3D surfaces (meshes) out of CT/MRI data using MorphoDig ?}
		Currently, the only possibility to produce a surface with MorphoDig is, using a given threshold value, to create an isosurface (using the marching cubes algorithm, for instance). Segmentation tools will be added in future versions of this software.
		\section{Is there a CTRL-Z functionality ?}
		Yes, definitely. Most actions can be undone and redone. 